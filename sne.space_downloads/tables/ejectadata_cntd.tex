
\begin{turnpage}
\tabletypesize{\tiny}
\begin{deluxetable*}{lccclccr}
  \singlespace
\setlength{\tabcolsep}{6pt} 
\centering
\tablewidth{614.88pt}
\tablecaption{Data suitable for extraction of ejecta properties, continued}
\tablehead{ & 
  \colhead{SN Type}  &
  \colhead{GRB}  &
\colhead{$z$ \tablenotemark{a}}&
\colhead{$E(B-V)_\mathrm{MW}$\tablenotemark{b}} &
\colhead{Host Galaxy} &
\colhead{$E(B-V)_\mathrm{host}$\tablenotemark{c}} &
\colhead {extinction reference}
}

\startdata
%\tabletypesize{\scriptsize}
\enddata
\tablenotetext{9}{\citet{folatelli06} cites $\naew \sim 1 \AA$, and derived an $E(B-V)_\mathrm{host} = 0.10$ value from the lower branch of the T03 relationship. All prescriptions we are using in our work suggest higher extinction, use the low extinction value adopted by \citet{folatelli06}.}

\tablenotetext{10}{High levels of reddening are noted and discussed in \citet{taubenberger05} and \citet{bianco14}.}

\tablenotetext{11}{\citet{campana06} reports  $E(B-V)_\mathrm{host} = 0.20$  by fitting the Rayleigh-Jeans tail of the blackbody emission at 32 ks (9 hours) assuming an SMC reddening law. We adjust that estimate to $E(B-V)_\mathrm{host} = 0.17$ assumig a MW reddening law, for consistency. \citet{guenther06} Na I D2 measures of the equivalent width are averaged after applying P12's prescription.} 

\tablenotetext{12}{\citet{blondin07} reports  $\naew \sim 3 \AA$, from which we assign $E(B-V)_\mathrm{host} = 0.48$ from T03's prescription.}

\tablenotetext{13}{\citet{blondin07b} reports a reddened spectrum, but no measure of $\naew$ due to the low SNR of the data.}

\tablenotetext{14}{\citet{valenti08} reports $\naew \sim 0.13 \AA$. We adjust their estimate of $E(B-V)_\mathrm{host} = 0.03$ to $E(B-V)_\mathrm{host} = 0.02$ following the P12 prescription.}

\tablenotetext{15}{\citet{chornock08} reports  $\naew \sim 1.8 \AA$, and derives  $E(B-V)_\mathrm{host} = 0.5$. In accordance to the T03 prescription, and in concert with \citet{pastorello08c}, we derive  $E(B-V)_\mathrm{host} = 0.29$ from the $\naew$ measured by \citet{chornock08}. Also, \citet{blondin08} reports reddening from correction of SN~2008ax to spectra of SN~1987A as $E(B-V)_\mathrm{host} = 0.6$. }
  
%\tablenotetext{16}{Radio relativistic, no GRB detected.}

\tablenotetext{18}{\citet{oates12} also reports an extinction $E(B-V)_\mathrm{host} \sim 0.18$ derived from correcting the spectra to spectra of other SN IIb, and adopt the average of the two extinction estimates. In order to not to bias ourselves and supress diversity we do not adopt spectra correction estimates.}



\tablenotetext{20}{\citet{follatelli14} reports extinction estimates from various methods, including report $\naew \sim 2.1\AA$, which would leads us an estimate of $E(B-V)_\mathrm{host} \sim 0.34$ using the T03 conversion, Balmer decrement, whih leads to  $E(B-V)_\mathrm{host} \sim 0.44$, and values obtained from spectral correction that are also consistent with the latter.}

\tablenotetext{21}{\citet{valenti12} reports the extinction derived from $\naew$ but not the $\naew$ or which conversion prescription is used. }


\tablenotetext{22}{\citet{kumar13} measures $\naew \sim 0.35\pm 0.29\AA$ from unresolved Na I D lines, thus this estimate is unreliable (P12) while \citet{moralesgaroffolo15} adopts the relationship in P12.}

\tablenotetext{23}{We adopt the value derived by \citet{bufano} from the blended Na I D absorption, but include both errors in the measurement, as reported by \citet{bufano},  and in the P12 prescription.}


\tablenotetext{24}{Xu measures $E(B-V)_\mathrm{host} \sim 0.03\pm0.01$ from the $\naew$ through the P12 conversion, but then adopts a slightly higher value of $E(B-V)_\mathrm{host} \sim 0.05$. For consistency we use the value of $E(B-V)_\mathrm{host} \sim 0.03\pm0.01$ obtained from the $\naew$ measurements through P12.}

\tablenotetext{25}{We use the Na I D measurement reported by \citet{walker14} $\naew \sim 0.15\pm0.12\AA$, but adopt P12's conversion.}

\tablenotetext{26}{We use the Na I D measurement reported by \citet{bernsten14} $\naew \sim 0.5 \AA$, but adopt P12's conversion. Note that \citet{bernsten14} derives a higher extinction value of $E(B-V)_\mathrm{host} \sim 0.17 \pm 0.03$ by spectral correction, but we ignore this value in order to not to bias ourselves agains the diversity of stripped SN.}
 
\end{deluxetable*}
\end{turnpage}
