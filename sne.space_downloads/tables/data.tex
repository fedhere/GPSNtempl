\begin{deluxetable*}{lccllcll}
\singlespace
\setlength{\tabcolsep}{0.0001in} 
\tablewidth{514.88pt}
\tablecaption{Discovery and Classification Data for SN Sample}
\tablehead{\colhead{SN Name}  & 
\colhead{CfA spectra\tablenotemark{a}}&
\colhead{CfA NIR\tablenotemark{b}}&
\colhead{RA}&
\colhead{Dec}& 
\colhead{SN Type\tablenotemark{c}}  &  
\colhead{Discovery Date}  &  
\colhead{Discovery Reference} } 
\startdata
{SN~2001ej}                     &M14&                           &7:23:43 &+33:26:38.0 &		Ib		&	2001-09-17	&	IAUC 7719	  \\
{SN~2001gd}\tablenotemark{3} 	&M14&                           &13:13:23& +36:38:17.7&		IIb		&	2001-11-24	&	IAUC 7761	  \\
{SN~2002ap}\tablenotemark{1}	&M14&                           &1:36:23 &+15:45:13.2 &		Ic-bl		&	2002-01-29	&	IAUC 7810	 \\
{SN~2003jd}\tablenotemark{1}    &M14&                           &21:03:38& -4:53:45   &		Ic-bl		&	2003-10-25	&	IAUC 8232	  \\
{SN~2004ao}\tablenotemark{1} 	&M14&	                        &17:28:09& +07:24:55.5&		Ib		&	2004-03-07	&	IAUC 8299	  \\
{SN~2004aw}\tablenotemark{1} 	&M14&                           &11:57:50& +25:15:55.1&		Ic		&	2004-03-19	&	IAUC 8310 	          \\
{SN~2004fe}			&M14&    	                &0:30:11 &+02:05:23.5 &		Ic		&	2004-10-30	&	IAUC 8425	  \\
{SN~2004gk}\tablenotemark{1,2} &M14&Y                           &12:25:33& +12:15:39.9&		Ic		&	2004-11-25	&	IAUC 8446	  \\
{SN~2004gq}\tablenotemark{2,3}  &M14&Y 	                        &5:12:04 &-15:40:54   &		Ib		&	2004-12-11	&	IAUC 8452	 \\ 
{SN~2004gt}\tablenotemark{2,4}  &M14&Y 	                        &12:01:50& -18:52:12  &		Ic		&	2004-12-12	&	IAUC 8454	  \\
{SN~2004gv}\tablenotemark{2}    &M14&                           &2:13:37&-0:43:05.8   &	        Ib 	        &       2004-12-13      &       IAUC 8454	  \\
{SN~2005az}\tablenotemark{2}    &M14&Y	                        &13:05:46& +27:44:08.4&		Ic		&	2005-03-28	&	IAUC 8503	  \\
{SN~2005bf}\tablenotemark{1}    &M14&Y	                        &10:23:57& -3:11:28   &		Ib		&	2005-04-06	&	IAUC 8507	    \\ 
{SN~2005ek}\tablenotemark{1}    &M14&Y$^*$                      &3:05:48 &+36:46:10&            Ic              &       2005-09-24      &       IAUC 8604        \\
% SN~2005eo}$^+$			&7:59:13 &+32:55:19.7 &		Ic		&	2005-09-27	&	IAUC 8605	& IAUC 8605     \\
{SN~2005hg}\tablenotemark{2 }   &M14&Y	                        &1:55:41 &+46:47:47.4 &		Ib		&	2005-10-25	&	IAUC 8623	 \\
{SN~2005kf}			&M14&             	        &7:47:26 &+26:55:32.4 &		Ic		&	2005-11-11	&	IAUC 8630	 \\
{SN~2005kl}			&M14&Y                          &12:24:35& +39:23:03.5&		Ic		&	2005-11-22	&	CBET 300	  \\
SN~2005kz\tablenotemark{2}	&&	                        &19:00:49& +50:53:01.8&		Ic		&	2005-12-01	&	IAUC 8639	  \\
{SN~2005la}\tablenotemark{1,2,b}&M14&	                        &12:52:15& +27:31:52.5&	        Ib-n/IIb-n      &	2005-11-30	&	IAUC 8639	  \\
{SN~2005mf}\tablenotemark{2}    &M14&Y	                        &9:08:42 &+44:48:51.4 &		Ic		&	2005-12-25	&	IAUC 8648	  \\
{SN~2005nb}\tablenotemark{2}	&M14&                           &12:13:37& +16:07:16.2&		Ic-bl		&	2005-12-17	&	CBET 357	  \\
SN~2006F\tablenotemark{2}	&&                              &2:28:11 &+19:36:13   &		Ib		&	2006-01-11	&	CBET 364	 \\
{SN~2006T}                      &M14&				&9:54:30 &-25:42:29   &		IIb		&	2006-01-30	&	IAUC 8666	  \\
{SN~2006aj}\tablenotemark{1}    &M14&Y	                        &3:21:39 &+16:52:02.6 &		Ic-bl		&	2006-02-18	&	IAUC 8674	 \\
SN~2006ba			&&	                        &9:43:13 &-9:36:53    &		IIb		&	2006-03-19	&	IAUC 8693	 \\
SN~2006bf			&&	                        &12:58:50&+9:39:30    &		IIb		&	2006-03-19	&	IAUC 8693	 \\
SN~2006cb			&&	                        &14:16:31&+39:35:15   &		Ib		&	2006-05-05	&	IAUC 8709	 \\
{SN~2006ck}\tablenotemark{2}	&M14&                           &13:09:40& -1:02:57   &		Ic		&	2006-05-20	&	IAUC 8713	 \\
{SN~2006el}\tablenotemark{2}	&M14&                           &22:47:38& +39:52:27.6&		Ib		&	2006-08-25	&	IAUC 8741	 \\
{SN~2006ep}			&M14&                           &0:41:24 &+25:29:46.7 &		Ib		&	2006-08-30	&	IAUC 8744	 \\
{SN~2006fo}\tablenotemark{2}    &M14&Y                          &2:32:38 &+00:37:03.0 &		Ib		&	2006-09-16	&	IAUC 8750	  \\
SN~2006gi\tablenotemark{1} 	&&	                        &10:16:46& 73:26:26   &		Ib		&	2006-09-18	&	CBET 630	 \\
SN~2006ir			&&	                        &23:04:35& 7:36:21    &		Ic		&	2006-09-24	&	CBET 658	 \\
{SN~2006jc}\tablenotemark{1,2,c} &M14&Y	                        &9:17:20 &+41:54:32.7 &		Ib-n		&	2006-10-09	&	IAUC 8762	 \\
{SN~2006lc}			&M14&	                        &22:44:24& -0:09:53   &		Ib		&	2006-10-25	&	CBET 693	 \\
{SN~2006ld}			&M14&Y                          &0:35:27 &+02:55:50.7 &		Ib		&	2006-10-19	&	IAUC 8766	 \\
{SN~2006lv}			&M14&	                        &11:32:03& +36:42:03.6&		Ib		&	2006-10-28	&	IAUC 8771	 \\
%006ss				&14:20:27& 35:11:42   &		IIb		&	2006-12-17	&	IAUC 8789	& CBAT 781     \\
{SN~2007C}\tablenotemark{2 }    &M14&Y                          &13:08:49& -6:47:01   &		Ib		&	2007-01-07	&	IAUC 8792	 \\
{SN~2007D}\tablenotemark{2 }    &M14&Y	                        &3:18:38 &+37:36:26.4 &		Ic-bl		&	2007-01-07	&	IAUC 8794	   \\ 
{SN~2007I}			&M14&Y                          &11:59:13& -1:36:18   &		Ic-bl		&	2007-01-14	&	IAUC 8798	 \\
{SN~2007ag}			&M14&	                        &10:01:35& +21:36:42.0&		Ib		&	2007-03-07	&	IAUC 8822	 \\
SN~2007aw			&&	                        &9:57:24 &-19:21:23   &		Ic		&	2006-03-22	&	IAUC 8829	  \\ 
{SN~2007bg}\tablenotemark{3,4} 	&M14&                           &11:49:26& +51:49:28.8&		Ic-bl		&	2007-04-16	&	IAUC 8834	           \\
{SN~2007ce}			&M14&Y                          &12:10:17& +48:43:31.5&		Ic-bl		&	2007-05-04	&	IAUC 8843	  \\
{SN~2007cl}			&M14&	                        &17:48:21& +54:09:05.2&		Ic		&	2007-05-23	&	IAUC 8851	 \\
{SN~2007gr}\tablenotemark{1}    &M14&Y	                        &2:43:27 &+37:20:44.7 &		Ic		&	2007-08-15	&	CBET 1034	  \\
{SN~2007hb}			&M14&	                        &2:08:34 &+29:14:14.3 &		Ic		&	2007-09-24	&	CBET 1043	    \\
{SN~2007iq}			&M14&	                        &6:13:32 &+69:43:49.2 &		Ic/Ic-bl	&	2007-09-12	&	CBET 1064	     \\
SN~2007ke\tablenotemark{1} 	&&	                        &2:54:23 &+41:34:16   &		Ib (Ca rich)	&	2007-09-25	&	CBET 1084	     \\
{SN~2007kj}			&M14&	                        &0:01:19 &+13:06:30.6 &		Ib		&	2007-10-02	&	CBET 1092	     \\
{SN~2007ru}\tablenotemark{1} 	&M14&                           &23:07:23& +43:35:33.7&		Ic-bl		&	2007-11-30	&	CBET 1149	     \\
%007rw				&12:38:03& -2:15:40.1 &  	IIb/IIL	        &	2007-12-04	&	CBAT 1154	& ASPC 451     \\
{SN~2007rz}\tablenotemark{6}	&M14&                           &4:31:10 &+07:37:51.5 &		Ic		&	2007-12-08	&	CBET 1158	     \\
{SN~2007uy}\tablenotemark{1}    &M14&Y	                        &9:09:35 &+33:07:08.9 &		Ib-pec		&	2007-12-31	&	IAUC 8908	     \\
{SN~2008D}\tablenotemark{1}     &M14&Y	                        &9:09:30 &+33:08:20.3 &		Ib		&	2008-01-07	&	GCN  7159	     \\
{SN~2008an}			&M14&	                        &17:38:28&+61:02:13.7b&		Ic		&	2008-02-24	&	CBAT 1268	      \\
{SN~2008aq}\tablenotemark{7}	&M14&                           &12:50:30& -10:52:01  &		IIb		&	2008-02-27	&	CBAT 1271	     \\
{SN~2008ax}\tablenotemark{1}    &M14&Y$^{*}$	                &12:30:40& +41:38:14.5&		IIb		&	2008-03-03	&	CBET 1280/6	     \\
{SN~2008bo}\tablenotemark{6}	&M14&                           &18:19:54& +74:34:20.9&		IIb		&	2008-04-01	&	CBET 1324	     \\
{SN~2008cw}			&M14&                           &16:32:38& +41:27:33.2&		IIb		&	2008-06-01	&	IAUC 1395	     \\
SN~2008hh			&&Y$^{*}$                        &1:26:03 &11:26:26   &		Ic		&	2008-11-20	&	CBET 1575 	        \\
SN~2009K			&&	                        &4:36:36 & -0:08:35.6 &         IIb		&	2009-01-14	&	CBET 1663 	     \\
{SN~2009er}\tablenotemark{f}    &M14&Y                          &15:39:29& +24:26:05.3&         Ib-pec		&	2009-05-22	&	CBAT 1811	    \\
{SN~2009iz}			&M14& Y                         &2:42:15 &+42:23:50.1 &		Ib		&	2009-09-19	&	CBET 1943	     \\
{SN~2009jf}\tablenotemark{1}    &M14&Y	                        &23:04:52& +12:19:59.5& 	Ib		&	2001-09-17	&	IAUC 7719	

\enddata
\tablenotetext{1}{Indicates SN whose early-time behavior have been studied in the literature. $^\mathrm{2}$ Included in the D11 sample. }
\tablenotetext{3}{Radio studies, no published optical lightcurve. $^4$Progenitor studies, no published lightcurve.$^5$Host studies, no published lightcurve.$^6$Only spectra published.$^7$Only UV data published.}    
\tablenotetext{a}{Object whose names are bold in the table have spectra presented in M14}
\tablenotetext{b}{Object for which NIR data is available within the CfA survey. $^\mathrm{*}$Indicates \emph{only} NIR data is available within the CfA survey for this object.}
\tablenotetext{c}{SN reclassified with SNID using updated templates in M14 have the new classification reported.} 
\tablenotetext{d}{SN~2005la is spectroscopically peculiar showing narrow He and H lines in emission (Pastorello 2008).}
\tablenotetext{e}{SN~2006jc is spectroscopically peculiar and  showed narrow He lines in emission. See text for details. }
\tablenotetext{f}{The type classification of SN 2009er remains ambiguous. SNID classifies SN 2009er differentlly at different epochs. We also note the presence of high velocity He I lines.}
\label{tab:sample}
\end{deluxetable*}
