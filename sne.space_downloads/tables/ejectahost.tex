%\begin{turnpage}
%\tabletypesize{\tiny}
\begin{deluxetable*}{lccccc}
  \singlespace
\setlength{\tabcolsep}{6pt} 
\centering
%\tablewidth{614.88pt}
\tablecaption{SN type and host galaxy details.}
\tablehead{ & 
  \colhead{SN Type}  &
  \colhead{GRB}  &
\colhead{$z$ \tablenotemark{a}}&
%\colhead{$E(B-V)_\mathrm{MW}$\tablenotemark{b}} &
\colhead{Host Galaxy} &
%\colhead{$E(B-V)_\mathrm{host} NaID$\tablenotemark{c}} &
%\colhead{$E(B-V)_\mathrm{host} env$\tablenotemark{c}} &
%\colhead{$E(B-V)_\mathrm{host} other$\tablenotemark{c}} &
%\colhead {extinction reference} & \colhead{NaID method}
}

\startdata
%\tabletypesize{\scriptsize}
% name		   &     type                      &   z                     &   host                \\
SN~1983V           &      Ic        &              &   0.0110\tm{1}            &  NGC~1365             \\
SN~1993J           &      IIb       &              &   -0.0001                  &  M81                  \\
SN~1994I           &      Ic        &              &   0.0015                  &  M51                  \\
SN~1996cb          &      IIb       &              &   0.0024                  &  NGC~3510             \\
SN~1998bw          &      Ic-BL     &980425        &   0.0085                  &  ESO~184-G82          \\
SN~1999dn          &      Ib        &              &   0.0093                  &  NGC~7714             \\
SN~1999ex          &      Ib        &              &   0.011                   &  IC5179               \\
SN~2002ap          &      Ic-BL     &              &   0.0022                  &  M74                  \\
SN~2003dh          &      Ic-BL     &030329        &   0.1685                  &  Anonimous            \\
SN~2003jd          &      Ic-BL     &              &   0.019                   &  MCG-01-59-21         \\
SN~2003lw          &      Ic-BL     &              &   0.1055                  &  Anonimous\tm{2}      \\
SN~2004aw          &      Ic        &              &   0.0159                  &  NGC~3997             \\
SN~2004fe          &      Ic        &              &   0.0179                  &  NGC~132              \\
SN~2004gq          &      Ib        &              &   0.0065\tm{1}            &  NGC~1832             \\
SN~2005az          &      Ic        &              &   0.0088                  &  NGC~4961             \\
SN~2005bf          &      Ib        &              &   0.0189                  &  NGC~4490             \\
SN~2005hg          &      Ib        &              &   0.0213\tm{1}            &  UGC 1394             \\
SN~2005kl          &      Ic        &              &   0.0035\tm{1}            &  NGC~4369             \\
SN~2006aj          &      Ic-BL     &060218        &   0.0334                  &  J032139.68+165201.7  \\
SN~2007C           &      Ib        &              &   0.0056\tm{1}            &  NGC~4981             \\          
SN~2007Y           &      Ib        &              &   0.0046                  &  NGC~1187             \\
SN~2007ag          &      Ib        &              &   0.0207                  &  UGC 05392            \\
SN~2007gr          &      Ic        &              &   0.0017                  &  NGC~1058             \\
SN~2007kj          &      Ib        &              &   0.0179                  &  NGC~7803             \\
SN~2007ru          &      Ic-BL     &              &   0.0155\tm{1}            &  UGC~12381            \\
SN~2008D           &      Ib        &              &   0.0070                  &  NGC~2770             \\
SN~2008ax          &      IIb       &              &   0.0019                  &  NGC~4490             \\
SN~2008bo          &      IIb       &              &   0.0050\tm{1}            &  NGC~6643             \\
SN~2009bb          &      Ic-BL     &undetected\tm{3}&   0.0099\tm{1}            &  NGC~3278             \\
SN~2009jf          &      Ib        &              &   0.0079                  &  NGC~7479             \\
SN~2009mg         &      IIb       &              &   0.0076                  &  ESO~121-26           \\
SN~2010ah\tm{4}    &      Ic-BL     &              &   0.0498                  &  Anonimous            \\
SN~2010as          &      IIb       &              &   0.0073                  &  NGC~6000             \\
SN~2010bh          &      Ic-BL     &100316D       &   0.0591                  &  Anonimous            \\
SN~2011bm          &      Ic        &              &   0.0015                  &  IC~3917              \\
SN~2011ei          &      IIb       &              &   0.0093\tm{1}            &  NGC~6925             \\
SN~2011fu          &      IIb       &              &   0.0185\tm{1}            &  UGC~1626             \\
SN~2011hs          &      IIb       &              &   0.0057\tm{1}            &  IC~5267              \\
SN~2012bz          &      Ic-BL     &              &   0.283                   &  Anonimous            \\
SN~2013cq          &      Ic-BL     &              &   0.3399                  &  J113232.84+274155.4  \\  
SN~2013df          &      IIb       &              &   0.0024                  &  NGC~4414             \\
SN~2013dx          &      Ic-BL     &              &   0.0145                  &  Anonimous            \\
PTF10qts           &      Ic-BL     & 130702A      &   0.0907                  &  J164137.53+28582      \\
iPTF13bvn          &      Ib        &              &   0.0045                  &  NGC~5806             

\enddata

\tablenotetext{a}{Redshift, as reported by Simbad (\url{http://simbad.u-strasbg.fr/simbad/}) for the SN or associated GRB.}


%\tablenotetext{e}{`Y': the photospheric velocity is measured from a single spectrum within 2 days of \maxep, `N': photospheric velocity is measured from two spectra stradling \maxep and within eight days of \maxep.}
\clearpage

\tablenotetext{1}{As reported by Simbad for the host galaxy.}

\tablenotetext{2}{\citet{prochaska04} suggets a possible identification of the host galaxy of SN~2003lw/GRB~031203 with HG 031203. The high line of sight extinction makes host galaxy identificatrion an characterization difficult.}

\tablenotetext{3}{Radio relativistic, no GRB detected.}
\tablenotetext{4}{Also PTF~2010bzf.}

\label{tab:hostgal}

 
\end{deluxetable*}
%\end{turnpage}
